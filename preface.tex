\chapter{Предисловие}

\section{Странная история этой книги}

В январе 1999 года я готовился преподавать вводные занятия по программированию
на Java. Я преподавал их уже три раза, и меня это не удовлетворяло. Показатель
результативности учеников был слишком низким и даже для успевающих студентов
общий уровень обучения был недостаточным.

Одной из проблем, которые я видел, были учебники. Они были слишком большими,
включали слишком много ненужных деталей о Java, и не имели достаточного
количества высокоуровневых рекомендаций по программированию в общем. И все они
страдали от эффекта люка: легкое начало, последовательное изложение материала, а
затем где-то на пятой главе пол проваливается. Студенты получают слишком много
нового материала, и слишком быстро, а я провожу остаток семестра собирая
осколки.

За две недели до начала занятий я решил написать свою собственную книгу.
Моими целями были:

\begin{itemize}
\item Будьте кратки. Для студентов лучше прочитать 10 страниц, чем не читать 50.
\item Будьте осторожны со словарем. Я старался минимизировать жаргон, и
определять каждый термин при первом использовании.
\item Вводите постепенно. Чтобы избежать люков, я взял самые сложные темы, и
разделил их на ряд небольших шагов.
\item Сосредоточьтесь на программировании, а не на языке программирования.
Я включил минимальное полезное подмножество Java и выкинул остальное.
\end{itemize}

Мне было нужно название, и по наитию я выбрал \emph{How to Think Like a Computer
Scientist} (думаем как программист).

Моя первая версия была грубой, но она работала. Студенты почитали, и поняли
достаточно, чтобы я мог потратить учебное время на трудные темы, на интересные
темы и (важнее всего) на практику.

Я выпустил книгу под лицензией GNU Free Documentation License, которая позволяет
пользователям копировать, изменять и распространять книгу.

Круче всего то, что произошло дальше. Джефф Элкнер (Jeff Elkner), преподаватель
ВУЗа в Виргинии, одобрил мою книгу и перевел её на Python. Он прислал мне копию
своей адаптации, и у меня появился необычный опыт изучения Python чтением моей
собственной книги. Я опубликовал первую Python-версию в издательстве Green Tea
Press в 2001 году.

В 2003 году я начал преподавать в Olin College\footnote{Franklin W.
Olin College of Engineering,  Needham, Массачусетс (около Бостона)}, и там я
впервые стал преподавать Python. Контраст с Java был поразительным. Студенты
меньше продирались через материал, учились большему, работали над более
интересными проектами, и в целом было гораздо веселее. 

Последние девять лет я продолжил развивать эту книгу, исправляя ошибки, улучшая
некоторые примеры и добавляя материал, особенно упражнения.

Результат\ --- эта книга, теперь с менее помпезным названием: \emph{Think
Python} (Думаем на Python). Вот некоторые из этих изменений:

\begin{itemize}
\item Я добавил раздел об отладке в конце каждой главы. Эти разделы показывают
общие методы поиска и предотвращения ошибок, и предупреждают о ловушках в
Python.
\item Я добавил больше упражнений, варьирующихся от коротких тестов до
нескольких значительных проектов. И я написал решения для большинства из них.
\item Я добавил серию учебных примеров\ --- более длинных заданий с
упражнениями, решениями и обсуждением. Некоторые из них основаны на Swampy,
наборе программ на Python, которые я написал для использования на моих занятиях.
Swampy, примеры кода, и некоторые решения доступны на
\url{http://thinkpython.com}.
\item Я расширил обсуждение program development plans и базовых шаблонов
проектирования.
\item Я добавил приложения об отладке, анализе алгоритмов, и диаграммами UML с
Lumpy.
\end{itemize}

\bigskip
Я надеюсь, что вы насладитесь работой с этой книгой, и она поможет вам научиться
программировать, и, хотя бы немного, думать как программист.

\bigskip
Аллен Б. Дауни 

Needham MA 

\bigskip
Аллен Дауни\ --- Профессор Информатики в Инженерном Колледже Франклина В. Олина.

\section{Благодарности}

Большое спасибо Джеффу Элкнеру, тому кто перевел мою книгу с Java на Python,
создал этот проект, и приобщил меня к тому, что, как оказалось, является моим
любимым языком программирования.
\bigskip

Спасибо также Крису Мейерсу, добавившему несколько разделов к \emph{How to Think
Like a Computer Scientist}.
\bigskip

Спасибо Фонду Свободного программного Обеспечения за разработку Свободной
Лицензии GNU на Документацию, которая помогла сделать моё сотрудничество с
Джеффом и Крисом возможным, и Creative Commons, за лицензию, которую я использую
сейчас.
\bigskip

Благодаря редакторов в Лулу, которые работали над \emph{How to Think Like a
Computer Scientist}.
\bigskip

Спасибо всем студентам, которые работали с ранними версиями этой книги, и всем
участникам (перечислены ниже), которые присылали корректировки и предложения.
\bigskip

\section{Список участников}

Более сотни зорких и вдумчивых читателей присылали мне предложения и исправления
в течение нескольких последних лет. Их вклад в этот проект и энтузиазм к нему
были огромной помощью.

Если у вас тоже есть предложения или исправления, пожалуйста, отправьте их на
\email{feedback@thinkpython.com}. Если я внесу изменение по вашим отзывам, я
добавлю вас в список участников (если вы не попросите, чтобы вас не учитывали).
 
Если вы включите по крайней мере часть предложения, в котором появляется ошибка,
мне будет легче ее искать. Номера страниц и разделов тоже подходят, но с ними не
так удобно работать. Спасибо!

\begin{itemize}
\item Lloyd Hugh Allen sent in a correction to Section 8.4. 
\item Yvon Boulianne sent in a correction of a semantic error in Chapter 5. 
\item Fred Bremmer submitted a correction in Section 2.1. 
\item Jonah Cohen wrote the Perl scripts to convert the LaTeX source for this
book into beautiful HTML.
\item Michael Conlon sent in a grammar correction in Chapter 2 and an
improvement in style in Chapter 1, and he initiated discussion on the technical
aspects of interpreters.
\item Benoit Girard sent in a correction to a humorous mistake in Section 5.6. 
\item Courtney Gleason and Katherine Smith wrote horsebet.py, which was used as
a case study in an earlier version of the book. Their program can now be found
on the website.
\item Lee Harr submitted more corrections than we have room to list here, and
indeed he should be listed as one of the principal editors of the text.
\item James Kaylin is a student using the text. He has submitted numerous
corrections.
\item David Kershaw fixed the broken catTwic. 
function in Section 3.10. 
\item Eddie Lam has sent in numerous corrections to Chapters 1, 2, and 3. He also fixed the Makefile 
so that it creates an index the first time it is run and helped us set up a versioning scheme. 
\item Man-Yong Lee sent in a correction to the example code in Section 2.4. 
\item David Mayo pointed out that the word ``unconsciously'' in Chapter 1 needed
to be changed to ``subconsciously''. 
\item Chris McAloon sent in several corrections to Sections 3.9 and 3.10. 
\item Matthew J. Moelter has been a long-time contributor who sent in numerous corrections and 
suggestions to the book. 
\item Simon Dicon Montford reported a missing function definition and several typos in Chapter 3. 
He also found errors in the incremen. 
function in Chapter 13. 
\item John Ouzts corrected the definition of "return value" in Chapter 3. 
\item Kevin Parks sent in valuable comments and suggestions as to how to improve the distribution 
of the book. 
\item David Pool sent in a typo in the glossary of Chapter 1, as well as kind words of encouragement. 
\item Michael Schmitt sent in a correction to the chapter on files and exceptions. 
\item Robin Shaw pointed out an error in Section 13.1, where the printTime function was used in an 
example without being defined. 
\item Paul Sleigh found an error in Chapter 7 and a bug in Jonah Cohen"s Perl script that generates 
HTML from LaTeX. 
\item Craig T. Snydal is testing the text in a course at Drew University. 
He has contributed several 
valuable suggestions and corrections. 
\item Ian Thomas and his students are using the text in a programming course. They are the first ones 
to test the chapters in the latter half of the book, and they have made numerous corrections and 
suggestions. 
\item Keith Verheyden sent in a correction in Chapter 3. 
\item Peter Winstanley let us know about a longstanding error in our Latin in Chapter 3. 
\item Chris Wrobel made corrections to the code in the chapter on file I/O and exceptions. 
\item Moshe Zadka has made invaluable contributions to this project. In addition to writing the first 
draft of the chapter on Dictionaries, he provided continual guidance in the early stages of the 
book. 
\item Christoph Zwerschke sent several corrections and pedagogic suggestions, and explained the 
difference between gleich and selbe. 
\item James Mayer sent us a whole slew of spelling and typographical errors, including two in the 
contributor list. 
\item Hayden McAfee caught a potentially confusing inconsistency between two examples. 
\item Angel Arnal is part of an international team of translators working on the Spanish version of 
the text. He has also found several errors in the English version. 
\item Tauhidul Hoque and Lex Berezhny created the illustrations in Chapter 1 and improved many 
of the other illustrations. 
\item Dr. Michele Alzetta caught an error in Chapter 8 and sent some interesting
pedagogic comments and suggestions about Fibonacci and Old Maid.
\item Andy Mitchell caught a typo in Chapter 1 and a broken example in Chapter 2. 
\item Kalin Harvey suggested a clarification in Chapter 7 and caught some typos. 
\item Christopher P. Smith caught several typos and helped us update the book for Python 2.2. 
\item David Hutchins caught a typo in the Foreword. 
\item Gregor Lingl is teaching Python at a high school in Vienna, Austria. He is
working on a German translation of the book, and he caught a couple of bad
errors in Chapter 5.
\item Julie Peters caught a typo in the Preface. 
\item Florin Oprina sent in an improvement in makeTime, a correction in printTime, and a nice typo. 
\item D. J. Webre suggested a clarification in Chapter 3. 
\item Ken found a fistful of errors in Chapters 8, 9 and 11. 
\item Ivo Wever caught a typo in Chapter 5 and suggested a clarification in Chapter 3. 
\item Curtis Yanko suggested a clarification in Chapter 2. 
\item Ben Logan sent in a number of typos and problems with translating the book into HTML. 
\item Jason Armstrong saw the missing word in Chapter 2. 
\item Louis Cordier noticed a spot in Chapter 16 where the code didn"t match the text. 
\item Brian Cain suggested several clarifications in Chapters 2 and 3. 
\item Rob Black sent in a passel of corrections, including some changes for Python 2.2. 
\item Jean-Philippe Rey at Ecole Centrale Paris sent a number of patches, including some updates 
for Python 2.2 and other thoughtful improvements. 
\item Jason Mader at George Washington University made a number of useful suggestions and corrections. 
\item Jan Gundtofte-Bruun reminded us that "a error
\item is an error. 
\item Abel David and Alexis Dinno reminded us that the plural of "matrix
\item is "matrices", not "matrixes". 
This error was in the book for years, but two readers with the same initials reported it 
on the same day. Weird. 
\item Charles Thayer encouraged us to get rid of the semi-colons we had put at the ends of some 
statements and to clean up our use of "argument
\item and "parameter". 
\item Roger Sperberg pointed out a twisted piece of logic in Chapter 3. 
\item Sam Bull pointed out a confusing paragraph in Chapter 2. 
\item Andrew Cheung pointed out two instances of "use before def."
\item C. Corey Capel spotted the missing word in the Third Theorem of Debugging and a typo in 
Chapter 4. 
\item Alessandra helped clear up some Turtle confusion. 
\item Wim Champagne found a brain-o in a dictionary example. 
\item Douglas Wright pointed out a problem with floor division in arc. 
\item Jared Spindor found some jetsam at the end of a sentence. 
\item Lin Peiheng sent a number of very helpful suggestions. 
\item Ray Hagtvedt sent in two errors and a not-quite-error. 
\item Torsten Hubsch pointed out an inconsistency in Swampy. 
\item Inga Petuhhov corrected an example in Chapter 14. 
\item Arne Babenhauserheide sent several helpful corrections. 
\item Mark E. Casida is is good at spotting repeated words. 
\item Scott Tyler filled in a that was missing. And then sent in a heap of corrections. 
\item Gordon Shephard sent in several corrections, all in separate emails. 
\item Andrew Turner spotted an error in Chapter 8. 
\item Adam Hobart fixed a problem with floor division in arc. 
\item Daryl Hammond and Sarah Zimmerman pointed out that I served up math.p. 
too early. And Zim spotted a typo. 
\item George Sass found a bug in a Debugging section. 
\item Brian Bingham suggested Exercise 11.10. 
\item Leah Engelbert-Fenton pointed out that I used tupl. 
as a variable name, contrary to my own 
advice. And then found a bunch of typos and a "use before def."
\item Joe Funke spotted a typo. 
\item Chao-chao Chen found an inconsistency in the Fibonacci example. 
\item Jeff Paine knows the difference between space and spam. 
\item Lubos Pintes sent in a typo. 
\item Gregg Lind and Abigail Heithoff suggested Exercise 14.4. 
\item Max Hailperin has sent in a number of corrections and suggestions. Max is one of the authors 
of the extraordinary Concrete Abstractions, which you might want to read when you are done 
with this book. 
\item Chotipat Pornavalai found an error in an error message. 
\item Stanislaw Antol sent a list of very helpful suggestions. 
\item Eric Pashman sent a number of corrections for Chapters 4"11. 
\item Miguel Azevedo found some typos. 
\item Jianhua Liu sent in a long list of corrections. 
\item Nick King found a missing word. 
\item Martin Zuther sent a long list of suggestions. 
\item Adam Zimmerman found an inconsistency in my instance of an "instance"\
and several other errors. 
\item Ratnakar Tiwari suggested a footnote explaining degenerate triangles. 
\item Anurag Goel suggested another solution for is\_abecedaria.
and sent some additional corrections. And he knows how to spell Jane Austen.
\item Kelli Kratzer spotted one of the typos. 
\item Mark Griffiths pointed out a confusing example in Chapter 3. 
\item Roydan Ongie found an error in my Newton"s method. 
\item Patryk Wolowiec helped me with a problem in the HTML version. 
\item Mark Chonofsky told me about a new keyword in Python 3. 
\item Russell Coleman helped me with my geometry. 
\item Wei Huang spotted several typographical errors. 
\item Karen Barber spotted the the oldest typo in the book. 
\item Nam Nguyen found a typo and pointed out that I used the Decorator pattern
but didn't mention it by name. 
\item Stephane Morin sent in several corrections and suggestions. 
\item Paul Stoop corrected a typo in uses\_only. 
\item Eric Bronner pointed out a confusion in the discussion of the order of operations. 
\item Alexandros Gezerlis set a new standard for the number and quality of suggestions he submitted. 
We are deeply grateful! 
\item Gray Thomas knows his right from his left. 
\item Giovanni Escobar Sosa sent a long list of corrections and suggestions. 
\item Alix Etienne fixed one of the URLs. 
\item Kuang He found a typo. 
\item Daniel Neilson corrected an error about the order of operations. 
\item Will McGinnis pointed out that polylin. 
was defined differently in two places. 
\item Swarup Sahoo spotted a missing semi-colon. 
\item Frank Hecker pointed out an exercise that was under-specified, and some broken links. 
\item Animesh B helped me clean up a confusing example. 
\item Martin Caspersen found two round-off errors. 
\item Gregor Ulm sent several corrections and suggestions. 
\item Dimitrios Tsirigkas suggested I clarify an exercise. 
\item Carlos Tafur sent a page of corrections and suggestions. 
\item Martin Nordsletten found a bug in an exercise solution. 
\item Lars O.D. Christensen found a broken reference. 
\item Victor Simeone found a typo. 
\item Sven Hoexter pointed out that a variable named inpu. 
shadows a build-in function. 
\item Viet Le found a typo. 
\item Stephen Gregory pointed out the problem with cm. 
in Python 3. 
\item Matthew Shultz let me know about a broken link. 
\item Lokesh Kumar Makani let me know about some broken links and some changes in error messages. 
\item Ishwar Bhat corrected my statement of Fermat"s last theorem. 
\end{itemize}

